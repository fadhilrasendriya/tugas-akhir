%-----------------------------------------------------------------------------%
\chapter{\babDua}
\label{bab:2}
%-----------------------------------------------------------------------------%
Untuk memulai penelitian, dibutuhkan kerangka berpikir yang sesuai untuk permasalahan yang ingin dipecahkan. Untuk membentuk kerangka berpikir yang sesuai, perlu dikaitkan dengan hasil studi literatur yang telah dilakukan. Oleh karena itu, pada bab ini, akan dijelaskan hasil studi literatur yang telah dilakukan yang telah dikaitan dengan kerangka kerja untuk penelitian ini.

\subsection{Masalah Umum pada Aplikasi Berbasis \textit{Blockchain}}
\label{sec:masalahblockchain}

Macrinici et.al., \citep{Macrinici2018} melakukan studi pemetaan sistematis terhadap 64 publikasi ilmiah tentang aplikasi \textit{smart contract}. Dalam penelitian tersebut diidentifikasi 16 masalah-masalah umum pada \textit{smart contract} yaitu mekanisme konsensus, \textit{sacrificed performance for scalability}, \textit{unpredictable state}, \textit{generating randomness}, \textit{timestamp dependency}, \textit{lack of reimbursement}, \textit{unilateral abortion}, \textit{lack of privacy}, \textit{call to the unknown}, \textit{exception disorder}, \textit{gasless send, out of gas exception}, \textit{typecast mismatch}, \textit{re-entrancy}, pemrograman \textit{smart contract}, \textit{stack overflow}, dan \textit{cryptocurrency transfer loss}. Masalah yang paling banyak timbul adalah \textit{unpredictable State, generating randomness,} dan pemrograman \textit{Smart Contract}. 

\subsection{Penelitian Terkait Sistem Logistik Berbasis \textit{Blockchain}}
\label{sec:penelitianlain}

Chang et.al., \citep{Chang2019} merancang ulang aplikasi \textit{supply chain tracking} berbasis \textit{smart contract} dan \textit{blockchain}. Aplikasi ini bertipe \textit{hybrid} dengan adanya akses ke data eksternal. Terdapat 6 \textit{smart contracts} pada aplikasi ini dengan 3 \textit{smart contract} untuk proses transaksi yaitu \textit{buyer contract}, \textit{supplier contract}, dan \textit{logistics contract}, \textit{payment contract} untuk proses pembayaran, dan 2 \textit{smart contract}, \textit{query forwarder contract} dan \textit{query dispatcher contract}, untuk mengakses data eksternal. Hasil analisis rancangan ulang aplikasi yaitu peningkatan pada 7 faktor efisiensi: \textit{tracability}, \textit{data storage}, \textit{privacy}, \textit{cost reduction}, \textit{cash liquidity}, \textit{payment}, dan \textit{degree of automation}.


Gao et.al., merancang sistem \textit{port supply chain} berbasis \textit{blockchain} Fabric bernama Fabric-PSChain \citep{Gao2022}. Terdapat enam \textit{user role} pada aplikasi ini yaitu \textit{consignor}, \textit{forwarder}, \textit{shipping company}, \textit{port}, \textit{consigee}, dan \textit{regulator}. Sistem menerapkan \textit{Role-Based Access Control Policy} (RBACP) agar \textit{user} terotentikasi sehingga informasi-informasi yang diterima \textit{user} sesuai berdasarkan \textit{role user} tersebut. Untuk mekanisme konsensus digunakan algoritma Kafka untuk mencegah pengunggahan informasi palsu.

