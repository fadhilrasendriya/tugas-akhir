%-----------------------------------------------------------------------------%
\chapter{\babDua}
\label{bab:2}
%-----------------------------------------------------------------------------%
Untuk memulai penelitian, dibutuhkan kerangka berpikir yang sesuai untuk permasalahan yang ingin dipecahkan. Untuk membentuk kerangka berpikir yang sesuai, perlu dikaitkan dengan hasil studi literatur yang telah dilakukan. Oleh karena itu, pada bab ini, akan dijelaskan hasil studi literatur yang telah dilakukan yang telah dikaitan dengan kerangka kerja untuk penelitian ini.

\section{Masalah Umum pada Aplikasi Berbasis \textit{Blockchain}}
\label{sec:masalahblockchain}

Pembahasan pertama adalah permasalahan-permasalahan yang umum muncul pada aplikasi berbasis \textit{blockchain}. Macrinici et.al., \citep{Macrinici2018} melakukan studi pemetaan sistematis terhadap 64 publikasi ilmiah tentang aplikasi \textit{smart contract}. Metode penelitian dilakukan dengan mencari publikasi dari berbagai sumber, kemudian dilakukan penyaringan berdasarkan judul, hasil, dan kesimpulan. Dari publikasi tersebut kemudian dianalisis dan diklasifikasi, kemudian baru dilakukan pengambilan data dan pemetaan masalah.

Dalam penelitian tersebut diidentifikasi 16 masalah-masalah umum pada \textit{smart contract}. Masalah-masalah tersebut yaitu mekanisme konsensus, \textit{sacrificed performance for scalability}, \textit{unpredictable state}, \textit{generating randomness}, \textit{timestamp dependency}, \textit{lack of reimbursement}, \textit{unilateral abortion}, \textit{lack of privacy}, \textit{call to the unknown}, \textit{exception disorder}, \textit{gasless send, out of gas exception}, \textit{typecast mismatch}, \textit{re-entrancy}, pemrograman \textit{smart contract}, \textit{stack overflow}, dan \textit{cryptocurrency transfer loss}. Masalah yang paling banyak timbul adalah \textit{unpredictable State, generating randomness,} dan pemrograman \textit{Smart Contract}. 

\section{Perbandingan \textit{Blockchain} Hyperledger Fabric dengan Ethereum}
Pembahasan selanjutnya adalah perbandingan antara \textit{blockchain} Hyperledger Fabric dengan Ethereum. Mohammed et.al., membuat perbandingan antara \textit{blockchain} Hyperledger Fabric dengan Ethereum \citep{Mohammed2021}. Selain perbedaan aksesibilitas (\textit{public permissionless} dengan \textit{private permissioned}), Hyperledger Fabric memiliki metode konsensus yang berbeda dengan Ethereum. Terdapat 3 metode konsensus pada Fabric yaitu solo, kafka, dan raft, namun metode konsensus solo dan kafka sudah \textit{deprecated} sejak versi Fabric 2.0 dikarenakan konsensus raft memungkinkan sebuah \textit{orderer} yang terdistribusi, tidak perlu konfigurasi kafka, serta memungkinkannya untuk ke depannya menjadi \textit{byzantine-fault-tolerant} \citep{hyperledger}. Sementara untuk Ethereum konsensus menggunakan metode \textit{proof-of-work}. 

Perbedaan lainnya adalah tidak dibutuhkannya sebuah \textit{cryptocurrency} pada Hyperledger Fabric dalam menggunakan \textit{smart contract}. Ethereum memerlukan \textit{mining} untuk menghidupkan jaringan, tidak halnya pada Hyperledger Fabric. Kemudian untuk pengembangan \textit{smart contract}, Hyperledger Fabric dapat diimplementasi dengan bahasa pemrograman Java, Node.js, atau Go sedangkan \textit{smart contract} Ethereum menggunakan Solidity \citep{Mohammed2021}.

\section{Penelitian Terkait Sistem Logistik Berbasis \textit{Blockchain}}
\label{sec:penelitianlain}

Pada bagian ini, dibahas mengenai penelitian-penelitian terkait sistem logistik berbasis \textit{blockchain}. Penelitian yang pertama yaitu rancangan ulang aplikasi \textit{supply chain tracking} berbasis \textit{smart contract} dan \textit{blockchain} \citep{Chang2019}. Aplikasi ini bertipe \textit{hybrid} dengan adanya akses ke data eksternal. Terdapat 6 \textit{smart contracts} pada aplikasi ini dengan 3 \textit{smart contract} untuk proses transaksi yaitu \textit{buyer contract}, \textit{supplier contract}, dan \textit{logistics contract}, \textit{payment contract} untuk proses pembayaran, dan 2 \textit{smart contract}, \textit{query forwarder contract} dan \textit{query dispatcher contract}, untuk mengakses data eksternal. Hasil analisis rancangan ulang aplikasi yaitu peningkatan pada 7 faktor efisiensi: \textit{tracability}, \textit{data storage}, \textit{privacy}, \textit{cost reduction}, \textit{cash liquidity}, \textit{payment}, dan \textit{degree of automation}.

Penelitian selanjutnya adalah rancangan sistem \textit{port supply chain} berbasis \textit{blockchain} Fabric bernama Fabric-PSChain \citep{Gao2022}. Terdapat enam \textit{user role} pada aplikasi ini yaitu \textit{consignor}, \textit{forwarder}, \textit{shipping company}, \textit{port}, \textit{consigee}, dan \textit{regulator}. Sistem menerapkan \textit{Role-Based Access Control Policy} (RBACP) agar \textit{user} terotentikasi sehingga informasi-informasi yang diterima \textit{user} sesuai berdasarkan \textit{role user} tersebut. Konsensus \textit{node} mencegah pengunggahan informasi palsu dengan cara setengah atau lebih \textit{node} harus memvalidasi suatu informasi, jika tidak maka informasi tersebut tidak akan ditulis ke dalam \textit{ledger} sehingga hal ini mencegah perusahaan mengubah data sembarangan dan meningkatkan reliabilitas transaksi. sistem penyimpanan yang terdistribusi juga melindungi dari perusakan data tanpa wewenang dan meningkatkan integritas data.

Selanjutnya merupakan penelitian mengenai aplikasi dan arsitektur \textit{blockchain} untuk operasi dan manajemen logistik pelabuhan \citep{Ahmad2021}. Penelitian ini mengajukan dan membandingkan arsitektur sistem operasi pelabuhan menggunakan Hyperledger Fabric dengan Hyperledger Besu. Kedua platform \textit{blockchain} tersebut dapat menjaga kerahasiaan data dengan komunikasi secara privat. Fitur privasi dari arsitektur yang diajukan dapat mengatasi masalah ketika para \textit{stakeholder} tidak sepenuhnya mempercayai satu sama lain. Dikatakan juga bahwa proses transaksi dengan Hyperledger Fabric dan Hyperledger Besu lebih cepat diselesaikan dibandingkan dengan \textit{blockchain} publik.
