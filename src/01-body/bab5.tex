%-----------------------------------------------------------------------------%
\chapter{\babLima}
\label{bab:5}
%-----------------------------------------------------------------------------%
Pada bab ini dijelaskan hasil eksperimen yang dilakukan pada skenario simulasi pada subbab \ref{sec:simulation}. Dilakukan pembahasan dan analisis terhadap hasil pengujian berdasarkan skenario dan dihubungkan dengan aspek fungsionalitas,  \textit{authentication, access control} dan \textit{reliability}. Untuk ringkasan hasil eksperimen dapat dilihat pada Tabel \ref{tab:simulationres}.

\section{Hasil dan Analisis Fungsional \textit{Smart Contract}}
Pada bagian ini dibahas hasil simulasi yang dilakukan pada subbab \ref{sec:simulation}. Hasil simulasi fungsional pertama yaitu fungsi CreateGood() atau pembuatan barang dapat dilihat pada \lst~\ref{lst:creategoodlog}. Untuk log lengkapnya dapat dilihat langsung \textit{file} 8\_CreateGood.txt\footnote{https://github.com/fadhilrasendriya/fabric-window/blob/main/test-network/logs/8\_CreateGood.txt}. Pembuatan barang dilakukan pada tahap inisialisasi barang dan log diambil pada 2 barang pertama. Dapat dilihat pada log untuk setiap barang yang dibuat transaksi dicatat dalam \textit{block} dan memakan waktu 2 detik hingga transaksi ter-\textit{commit} dalam block. Hal ini dikarenakan dalam simulator, \textit{client} melakukan SubmitTransaction untuk membuat barang dipanggil secara sinkronus dan menunggu hasil hingga transaksi ter-\textit{commit} dalam \textit{block}.

%\subsection{Pembuatan Barang}
%\label{subsec:creategood-chap5}
 

\lstinputlisting[label={lst:creategoodlog}, caption=Log pembuatan barang]{assets/logs/8_CreateGood.txt}

Dapat dilihat bahwa untuk setiap pemanggilan SubmitTransaction, transaksi dipropagasi ke seluruh \textit{peer} yang terhubung dan tergabung dalam \textit{channel} singlewindow kemudian \textit{chaincode} dieksekusi pada setiap \textit{peer} untuk dilakukan \textit{endorse} transaksi. Hal ini dapat dilihat pada log peer0.tradingcompany1.example.com pada indeks \textbf{adcc} dan \textbf{add1}. Setelah mayoritas \textit{peer} selesai melakukan \textit{endorse}, transaksi dikirim ke \textit{orderer} untuk kemudian dilakukan \textit{event ordering} dan dibuatkan \textit{block}. Setelah \textit{block} selesai dibuat \textit{block} disebarkan ke seluruh peer anggota jaringan dan kemudian di-commit pada masing-masing \textit{peer}. Kemudian dapat dilihat untuk \textit{block} yang telah dibuat (\textit{block} 711 dan 712) memiliki nilai \textit{hash} yang sama di setiap \textit{peer}.


Selanjutnya adalah hasil pemanggilan fungsi GetGoodById() untuk melihat barang yang telah dibuat sebelumnya. Hasil dari simulasi ini dapat dilihat pada \lst~\ref{lst:readgoodlog} atau untuk log lengkapnya dapat dilihat pada \textit{file} 9\_ReadGood.txt\footnote{https://github.com/fadhilrasendriya/fabric-window/blob/main/test-network/logs/9\_ReadGood.txt}. Untuk membaca data dari \textit{world state}, digunakan \textit{API} EvaluateTransaction(). Berbeda dengan SubmitTransaction(), pemanggilan ini bersifat lokal dan hanya membaca melalui \textit{peer} yang dituju. Dapat dilihat bahwa hanya \textit{peer} peer0.singlewindow.example.com yang menerima transaksi EvaluateTransaction dan tidak dikirim ke \textit{orderer} sehingga tidak ada \textit{block} yang dibuat.

\lstinputlisting[label={lst:readgoodlog}, caption=Log pembacaan barang]{assets/logs/9_ReadGood.txt}

Hasil selanjutnya yaitu pembuatan \textit{order} pada pemanggilan fungsi CreateOrder(). \textit{Order} pertama adalah \textit{order} dengan spesifikasi barang yang tidak melebihi batas impor maupun ekspor. Hasil pembuatan \textit{order} dapat dilihat pada \lst~\ref{lst:createorderlog} atau pada \textit{file} 10\_CreateOrderSuccess.txt\footnote{https://github.com/fadhilrasendriya/fabric-window/blob/main/test-network/logs/10\_CreateOrderSuccess.txt} untuk log yang lebih detail. Terdapat log pada \textit{chaincode} yang menunjukkan alur pembuatan \textit{order}. Untuk log pada \textit{peer} dan \textit{orderer} serupa dengan log CreateGood() pada bagian sebelumnya yaitu berupa pemanggilan \textit{chaincode}, \textit{endorse}, \textit{submit} transaksi ke \textit{orderer}, dan \textit{commit}. Dapat dilihat bahwa \textit{block} berhasil dibuat dan transaksi berhasil dicatat.

\lstinputlisting[label={lst:createorderlog}, caption=Log pembuatan \textit{order}]{assets/logs/10_CreateOrderSuccess.txt}

Pembuatan \textit{order} yang kedua adalah pembuatan \textit{order} dengan kuantitas barang yang melebihi batas yang diperbolehkan. Hasil pengujian berupa log yang dapat dilihat pada \lst~\ref{lst:createorderfailedlog} atau \textit{file} 12\_CreateInvalidOrder.txt\footnote{https://github.com/fadhilrasendriya/fabric-window/blob/main/test-network/logs/12\_CreateInvalidOrder.txt} untuk detailnya. Pada log \textit{chaincode} dapat dilihat bahwa pembuatan gagal pada saat melakukan verifikasi terhadap kuantitas barang yang terhadap aturan pada barang tersebut. Karena transaksi tersebut mengembalikan \textit{error} transaksi tersebut tidak di-\textit{broadcast} ke \textit{peer} lain sehingga tidak di-endorse dan tidak dikirim ke \textit{orderer}. Dapat dilihat bahwa tidak ada log yang masuk ke \textit{peer} lainnya.

\lstinputlisting[label={lst:createorderfailedlog}, caption=Log pembuatan \textit{order} yang gagal]{assets/logs/12_CreateInvalidOrder.txt}

%\subsection{Membaca \textit{Order}}
Untuk memasikan \textit{order} telah berhasil dibuat dilakukan pengujian fungsionalitas fungsi ReadOrder() yaitu membaca \textit{order} yang sebelumnya dibuat. Hasil pembacaan \textit{order} dapat dilihat pada \lst~\ref{lst:readorderlog} atau \textit{file} 11\_ReadCreatedOrder.txt\footnote{https://github.com/fadhilrasendriya/fabric-window/blob/main/test-network/logs/11\_ReadCreatedOrder.txt}. Seperti halnya pada fungsi GetGoodById(), API yang digunakan adalah EvaluateTransaction() dan \textit{chaincode} dieksekusi hanya secara lokal (tidak melibatkan \textit{peer} lain.)

\lstinputlisting[label={lst:readorderlog}, caption=Log pembacaan \textit{order}]{assets/logs/11_ReadCreatedOrder.txt}

\section{Hasil dan Analisis Aspek \textit{Authentication}}
Hasil log evaluasi \textit{authentication} dapat dilihat pada \lst~\ref{lst:differentpeer} atau \textit{file} 16\_SubmitToOtherOrganizationPeer.txt\footnote{https://github.com/fadhilrasendriya/fabric-window/blob/main/test-network/logs/16\_SubmitToOtherOrganizationPeer.txt}. bahwa \textit{client} tidak terotorisasi dan ditolak untuk berkomunikasi melalui \textit{peer} milik organisasi singlewindow.example.com

\lstinputlisting[label={lst:differentpeer}, caption=\textit{Request} menggunakan \textit{peer} organisasi lain]{assets/logs/16_SubmitToOtherOrganizationPeer.txt}

\section{Hasil dan Analisis Aspek \textit{Access Control}}
Selanjutnya, untuk menguji aspek \textit{access control}, dilakukan pemanggilan fungsi CreateGood() yang telah didefinisikan pada tabel \ref{table:smartcontract} menggunakan identitas milik organisasi tradingcompany.example.com, yang seharusnya hanya dapat dipanggil oleh organisasi singlewindow.example.com. Hasil pengujian ini dapat dilihat pada \lst~\ref{lst:creategoodunauthorizedlog} atau \textit{file} 13\_CreateGoodUnauthorized.txt\footnote{https://github.com/fadhilrasendriya/fabric-window/blob/main/test-network/logs/13\_CreateGoodUnauthorized.txt}. Dapat dilihat pada log bahwa pemanggilan \textit{chaincode} gagal dan mengembalikan \textit{error} sehingga INSW tetap dapat mengatur barang-barang yang diperbolehkan tanpa diubah oleh pihak tidak berwenang.

\lstinputlisting[label={lst:creategoodunauthorizedlog}, caption=Membuat barang menggunakan identitas tidak terotorisasi]{assets/logs/13_CreateGoodUnauthorized.txt}

\section{Hasil dan Analisis Aspek \textit{Reliability}}
Selanjutnya, aspek \textit{reliability} diuji dengan mengirim \textit{request} secara \textit{concurrent}. Hasil pengujian untuk \textit{request concurrent} tanpa timeout dapat dilihat pada \lst~\ref{lst:concurrentnotimeout} atau \textit{file} 14\_ConcurrentOrderNoTimeout.txt\footnote{\scriptsize https://github.com/fadhilrasendriya/fabric-window/blob/main/test-network/logs/14\_ConcurrentOrderNoTimeout.txt}. Dapat dilihat bahwa sistem tetap mampu melayani \textit{request} hingga 10000 namun terjadi \textit{crash} saat melayani 100000 \textit{request}. 

\lstinputlisting[label={lst:concurrentnotimeout}, caption=\textit{Request concurrent} tanpa \textit{timeout client}]{assets/logs/14_ConcurrentOrderNoTimeout.txt}

Sedangkan untuk hasil uji \textit{request concurrent} dengan \textit{timeout} pada \textit{client} dapat dilihat pada \lst~\ref{lst:concurrentwithtimeout} atau \textit{file} 15\_ConcurrentOrderWithNormalTimeout.txt\footnote{\scriptsize https://github.com/fadhilrasendriya/fabric-window/blob/main/test-network/logs/15\_ConcurrentOrderWithNormalTimeout.txt}. Dalam 5 iterasi, 4 dari 5 iterasi mengalami kegagalan pada 4000 \textit{request} ke atas dan 1 dari 5 iterasi terdapat 8 \textit{request} gagal pada 3000 \textit{request}. Untuk durasi 9 request adalah sekitar 2 detik sedangkan untuk \textit{request} hanya berdurasi rata-rata 0.14 detik. Hal ini dikarenakan dalam konfigurasi \textit{orderer}, \textit{block} akan menampung transaksi dalam \textit{batch} 10 transaksi setiap \textit{block} dengan \textit{timeout} 2 detik. Dengan kata lain pembuatan \textit{block} akan ditunda hingga 2 detik jika jumlah transaksi kurang dari 10.

\lstinputlisting[label={lst:concurrentwithtimeout}, caption=\textit{Request concurrent} dengan \textit{timeout client}]{assets/logs/15_ConcurrentOrderWithNormalTimeout.txt}

%\subsection{Potensi Serangan \textit{Denial-of-Service}}
Jika dibandingkan antara dua skenario untuk \textit{client} dengan \textit{timeout} dan tanpa \textit{timeout}, \textit{client} dengan timeout akan menggagalkan \textit{request} yang sedang berlangsung jika menunggu terlalu lama untuk menyelesaikan tahap-tahapan transaksi. Namun ketika \textit{client} dikonfigurasi tanpa menggunakan \textit{timeout}, sistem tetap akan melayani \textit{request} hingga selesai dibuatkan \textit{block} sehingga untuk melakukan transaksi baru harus menunggu seluruh transaksi sebelumnya selesai diproses. Hal ini menimbulkan celah serangan yang membuat sistem mengalami \textit{denial-of-service} dengan cara melakukan \textit{spam request} dengan jumlah yang sangat banyak. Untuk mengatasi hal ini dapat diterapkan \textit{rate limiting} untuk setiap \textit{client} yang menggunakan sistem ini. Selain \textit{rate limiting} pada \textit{client} dapat juga diterapkan \textit{cryptocurrency} sebagai biaya transaksi.



\begin{center}
\begin{longtable}{p{2cm}p{7cm}p{4cm}}
\caption{Ringkasan hasil simulasi.} 
\label{tab:simulationres} \\

\hline \multicolumn{1}{c}{\textbf{Metrik simulasi}} & \multicolumn{1}{c}{\textbf{Hasil}} & \multicolumn{1}{c}{\textbf{\textit{Smart contract} terlibat}} \\
\newline \\ \hline 
\endfirsthead

\multicolumn{3}{c}%
{{\bfseries \tablename\ \thetable{} -- lanjutan dari halaman sebelumnya}} \\
\hline \multicolumn{1}{c}{\textbf{Metrik simulasi}} & \multicolumn{1}{c}{\textbf{Hasil}} & \multicolumn{1}{c}{\textbf{\textit{Smart contract} terlibat}} \\ \hline 
\endhead

\hline \multicolumn{3}{|r|}{{Dilanjutkan di halaman selanjutnya}} \\ \hline
\endfoot

\hline \hline
\endlastfoot

Fungsional & Fitur \textit{delivery order} berhasil diimplementasi. & (i) CreateGood() \newline (ii) GetGoodById() \newline (iii) CreateOrder() \newline (iv) ReadOrder() \\

\hline \\

\textit{Authentication} & Identitas asing tidak dapat mengakses sistem \textit{blockchain} & - \\

\hline \\

\textit{Access Control} & Pembatasan akses pada fungsi tertentu dapat dibatasi untuk organisasi tertentu saja. & (i) CreateGood() \\

\hline \\

\textit{Reliability} & Sistem mampu menangani hingga 10000 transaksi bersamaan namun membutuhkan waktu yang cukup lama (lebih dari 100 detik). Berpotensi menghalangi pengguna lain yang ingin mengakses sistem. Dapat diimplementasikan \textit{rate limiting} untuk mencegah serangan \textit{denial-of-service}. & (i) CreateOrder() \\

\end{longtable}
\end{center}