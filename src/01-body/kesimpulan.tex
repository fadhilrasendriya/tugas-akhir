%---------------------------------------------------------------
\chapter{\kesimpulan}
\label{bab:6}
%---------------------------------------------------------------
Pada bab ini, dijelaskan kesimpulan atas penelitian sistem \textit{delivery order} berbasis \textit{blockchain} Hyperledger Fabric. Terdapat juga saran untuk pengembang dan serta saran untuk penelitian selanjutnya.


%---------------------------------------------------------------
\section{Kesimpulan}
\label{sec:kesimpulan}
%---------------------------------------------------------------

Proses verifikasi manual pada pembuatan \textit{delivery order} dapat dihilangkan. Proses tersebut dapat diimplementasi menggunakan komponen \textit{chaincode} pada Hyperledger Fabric. Dengan adanya \textit{smart contract} yang menggantikan proses manual, kegiatan penyelewengan seperti penyuapan dapat menurun karena semakin sedikit celah untuk dapat dilakukan intervensi. 

Untuk mengimplementasi \textit{chaincode} digunakan \textit{library} fabric-contract-api sedangkan untuk mengaksesnya dibutuhkan suatu \textit{frontend} dan \textit{client} yang dapat diimplementasi menggunakan \textit{library} fabric-gateway. \textit{Library fabric-contract-api} memiliki antarmuka fungsi untuk membaca dan menyimpan data dari dan ke \textit{ledger}. Fitur membaca data pada \textit{library} ini (GetState()) dimanfaatkan pada logika pembuatan \textit{delivery order} pada \textit{smart contract} untuk verifikasi daftar barang yang akan dikirim.

Sementara untuk cara mengimplementasi proses \textit{delivery order} menggunakan bantuan Hyperledger Fabric SDK adalah, melengkapi yang telah disebutkan di paragraf sebelumnya, dengan membuat dua tipe penyimpanan data yaitu data barang dan data \textit{delivery order}. Verifikasi daftar barang yang akan dikirim dalam \textit{delivery order} dilakukan dengan membaca data barang-barang menggunakan fungsi GetState() untuk diverifikasi batasnya dengan kuantitas yang ingin dikirim. Setelah semua terverifikasi dan sah, data \textit{delivery order} disimpan ke \textit{ledger} dengan fungsi PutState() kemudian ditunggu hingga \textit{block} untuk transaksi tersebut sudah dibuat dan transaksi sudah di-\textit{commit}. Data \textit{order} yang telah di-\textit{commit} tersebut menandakan bahwa \textit{delivery order} telah berhasil dan selesai dibuat.

Keunggulan lain sistem \textit{delivery order} berbasis \textit{blockchain} dalam aspek \textit{authentication} yaitu sistem hanya dapat diakses oleh pihak yang sudah terdaftar dalam jaringan \textit{blockchain} sehingga sistem tidak dapat diakses oleh pihak tidak dikenal. Untuk aspek \textit{access control}, fitur tertentu masih dapat dibatasi hanya dapat diakses oleh organisasi tertentu, yang dalam hal ini pengaturan dan pengubahan tentang daftar barang untuk dikirim hanya dapat dilakukan oleh pihak INSW. Kemudian untuk aspek \textit{reliability} menurut eksperimen sistem dapat melayani hingga 2000 \textit{request} bersamaan tanpa terdapat transaksi yang gagal.

Melihat kembali pada celah potensi penyuapan pada sistem INSW saat ini, celah tersebut dapat dihilangkan dengan mengganti proses verifikasi manual menjadi bagian dari \textit{logic smart contract}. Solusi tersebut dapat juga diimplementasikan menggunakan sistem tradisional yang tersentralisasi dan menggunakan \textit{database}. Namun dikarenakan banyak organisasi luar yang akan menggunakan sistem ini, diperlukan tingkat transparansi serta keamanan yang lebih tinggi untuk meningkatkan kepercayaan pengguna sistem. Dalam hal ini pendekatan \textit{blockchain} memiliki keunggulan dalam hal transparansi dan keamanan dari karakteristik \textit{tamper-evident} dan \textit{tamper-resistant} (lihat Subbab \ref{sec:litblockchain}) pada \textit{blockchain} yang tidak dimiliki oleh sistem tradisional.

%---------------------------------------------------------------
\section{Saran}
\label{sec:saran}
%---------------------------------------------------------------
Berdasarkan hasil penelitian ini, terdapat saran untuk pengembang dan untuk penelitian selanjutnya. Dikarenakan sistem memiliki celah serangan \textit{spam} yang dapat mengakibatkan sistem \textit{denial-of-service}, saran pertama yaitu perlu diimplementasikannya \textit{rate limiting} untuk membatasi jumlah \textit{request} yang bisa diajukan oleh \textit{client}. Selain itu dapat juga diimplementasikan \textit{cryptocurrency} sebagai biaya transaksi . Saran kedua yaitu dapat dilakukan penelitian lebih lanjut untuk aspek efisiensi, \textit{scalability}, \textit{latency}, dan aspek lainnya.

Teknologi \textit{blockchain} masih memiliki potensi-potensi lain dalam sistem \textit{supply-chain} maupun sistem-sistem lainnya. Ditambah dengan meningkatnya \textit{trend} dan publikasi penelitian aplikasi \textit{blockchain} \citep{Macrinici2018} semakin mendekatnya teknologi \textit{blockchain} dapat diadaptasi lebih banyak. Penelitian ini diharapkan dapat membantu bagi pengembang aplikasi \textit{blockchain} maupun membantu untuk penelitian-penelitian \textit{blockchain} setelah ini.
