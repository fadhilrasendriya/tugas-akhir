%
% Halaman Abstrak
%
% @author  Andreas Febrian
% @version 2.1.2
% @edit by Ichlasul Affan
%

\chapter*{Abstrak}
\singlespacing

\vspace*{0.2cm}

% Untuk conditional statement pembimbing dua
\def\blank{}

\noindent \begin{tabular}{l l p{10cm}}
	Nama&: & \penulis \\
	Program Studi&: & \program \\
	Judul&: & \judul \\
	Pembimbing&: & \pembimbingSatu \\
	\ifx\blank\pembimbingDua
    \else
        \ &\ & \pembimbingDua \\
    \fi
    \ifx\blank\pembimbingTiga
    \else
    	\ &\ & \pembimbingTiga \\
    \fi
\end{tabular} \\

\vspace*{0.5cm}

\noindent Ekspor dan impor merupakan kegiatan jual-beli barang antarnegara yang memerlukan dokumen bernama \textit{delivery order} untuk mengirim barang. Di Indonesia kegiatan ini didukung oleh sistem \textit{single-window} yang dinamakan INSW atau Indonesia \textit{National Single Window} yang juga menangani pembuatan dokumen \textit{delivery order}. Saat ini, dokumen permohonan \textit{delivery order} perlu dilakukan verifikasi manual dua pihak yaitu INSW dan \textit{shipping line}. Terdapat peluang pemanfaatan teknologi \textit{blockchain} bernama \textit{smart contract} untuk mengotomasi proses pembuatan dokumen \textit{delivery order} tanpa memerlukannya verifikasi manual namun tetap memiliki persetujuan semua pihak terlibat. Penelitian merupakan simulasi proses bisnis \textit{delivery order} berbasis \textit{blockchain} dan \textit{smart contract}, kemudian dianalisis aspek fungsionalitas, \textit{authentication}, \textit{access control}, dan \textit{reliability}. Hasil dari simulasi yaitu proses bisnis \textit{delivery order} tanpa verifikasi manual dapat diimplementasi menggunakan \textit{blockchain} Hyperledger Fabric dengan tetap menjaga kepercayaan dan persetujuan semua pihak yang terlibat.\\

\vspace*{0.2cm}

\noindent Kata kunci: \\ Hyperledger Fabric, \textit{single-window}, \textit{blockchain}, \textit{delivery order}, \textit{access control}, \textit{failure rate} \\

\setstretch{1.4}
\newpage
