%
% Halaman Abstract
%
% @author  Andreas Febrian
% @version 2.1.2
% @edit by Ichlasul Affan
%

\chapter*{ABSTRACT}
\singlespacing

\vspace*{0.2cm}

% Untuk conditional statement pembimbing dua
\def\blank{}

\noindent \begin{tabular}{l l p{11.0cm}}
	Name&: & \penulis \\
	Study Program&: & \studyProgram \\
	Title&: & \judulInggris \\
	Counsellor&: & \pembimbingSatu \\
	\ifx\blank\pembimbingDua
	\else
		\ &\ & \pembimbingDua \\
	\fi
	\ifx\blank\pembimbingTiga
	\else
		\ &\ & \pembimbingTiga \\
	\fi
\end{tabular} \\

\vspace*{0.5cm}

\noindent International trade require a document called delivery order before sending cargo to the expedition. In Indonesia, the trading activities are supported by a single-window system called INSW or Indonesia National Single Window, which also handles the creation of those documents. Currently, the verification of the delivery order application needs to be manually performed by two parties: INSW and Shipping Line. There is an opportunity to utilize blockchain technology, specifically smart contracts, to automate the process of creating delivery order documents without the need for manual verification while still ensuring the approval of all involved parties. The research involves simulating the delivery order process based on blockchain and smart contracts, then analyzing aspects of functionality, authentication, access control, and reliability. The result of the simulation shows that the delivery order process without the manual verification can be implemented using the Hyperledger Fabric blockchain while maintaining the trust and approval of all parties. \\

\vspace*{0.2cm}

\noindent Key words: \\ Hyperledger Fabric, single-window, blockchain, delivery order, access control, failure rate \\

\setstretch{1.4}
\newpage
